\chapter{Introduction}
\label{ch:intro}
\tab{\Large This section gives a scope description and overview of everything included in this SRS document. Also the purpose of this document is described and a list of abbreviations and definitions is provided.}
\section{Purpose}
\tab{The purpose of this document is to give a detailed description of the requirements of the "“SHADOWS“" software (game).It will illustrate the purpose and complete declaration for the development of system. It will also explain constraints, interface and interactions with users and external applications. This document is primary intended to be proposed to a customer for its approval and a reference for describing the first version of the system for the development team.}
\section{Document Conventions}
%\begin{table}
	%\resziebox{\textwidth}{!}{%
	\begin{tabu} to 1.0\textwidth{|X[l]|X[r]|}
		\hline
		Term & Definitions\\
		\hline\hline
		Users/players & Someone who interacts with the system\\
		\hline
		Fps & First Person Shooter\\
		\hline
	\end{tabu}
%\end{table}
%\section{Intended Audience and Reading Suggestions}
\section{Product Scope}
\tab{"SHADOWS"” is a multiplayer, fps, horror game mainly designed for people with greater than 16 years .It is for entertainment purpose only. Players can join by lobby and play game.}
\section{References}
{\begin{enumerate}
	\item https://unity3d.com/learn/tutorials
	\item https://www.assetstore.unity3d.com/en/\#
\end{enumerate}}

\chapter{Overall Description}
\label{Overall Description}
\tab{\Large This section will give an overview of the whole system. The system will be explained in its context to show how the system interacts with another system and introduce the basic functionality of it. It will also describe what type of stakeholders that will use the system and what functionality is available for each type. At last, the constraints and assumptions for the system will be presented. }

\section{Product Perspective}

\section{Product Functions}
\tab{With this application from the menu driven GUI interface can play game or quit game. He can create a lobby with other players and play game using keyboard and mouse. KEY FEATURES OF THE GAME PLZ MENTION.}

\section{User Classes and Characteristics}
\tab{There are only one type of users in this system, the players. Players interact with the system with the help of GUI interface.}

\section{Operating Environment}

\section{Design and Implementation Constraints}
\tab{The network connection is a constraint for the application. Since the application fetches data from the database over the internet. It is crucial that there is an Internet connection for the application to function.}\\
\tab {Power failure is also a constraint, as there is a multiplayer game, so there is no saving option.}

\section{User Documentation}
\tab{This product is under development state and requires a complete implemented prototype to explain the user documentation.}

\section{Assumptions and Dependencies}
\tab{Our assumption about the product is that it will always be used on Windows OS and pc having enough capability to run this application smoothly.}\\
\tab{Another assumption is that there is always a good Internet connection.}

\newpage




\chapter{Specific Requirements}
\label{Specific Requirements}
\tab{\Large This section contains all of the functional and quality requirements of the system. It gives a detailed description of the system and all its features.}
\section{External Interface Requirements}
\tab{This section provides a detailed description of all inputs into and outputs from the system. It also gives a description of the hardware, software and communication interfaces and provides basic prototypes of the user interface.}
\subsection{User Interfaces}
\tab{RNB DO IT PHOTO LAGBE AR DESCRIPTION}
\subsection{Hardware Interfaces}
\tab{Since the application does not have any designated hardware. It does not have any direct hardware interface.}
\subsection{Software Interfaces}
\tab{It uses mostly C\# language for coding, Unity game engine for developing the game and Blender for graphical models.	}
%\section{Communications Interfaces}


%\chapter{System Features}
%\label{System Features}

%\section{System Feature 1}
\section{Functional Requirements}

\section{Other Nonfunctional Requirements}
%\label{Other Nonfunctional Requirements}

\subsection{Performance Requirements}
\subsection{Safety Requirements}
\subsection{Security Requirements}
%\section{Software Quality Attributes}
%\section{Business Rules}

\section{Other Requirements}
%\label{Other Requirements}
\subsection{Legal Requirements}

\begin{appendices}
\chapter{Glossary}
\chapter{Analysis Models}
\chapter{To Be Determined List}


\end{appendices}


