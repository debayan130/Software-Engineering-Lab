%% Documenclass 
\documentclass[a4paper,oneside,titlepage]{report}

%% Packages
\usepackage[english]{babel}
\usepackage{amsmath}
\usepackage{complexity}
\usepackage[T1]{fontenc}
\usepackage[utf8]{inputenc}
\usepackage{graphicx} %%Graphics in pdfLaTeX
\usepackage{a4wide} %%Smaller margins, more text per page.
\usepackage{longtable} %%For tables that exceed a page width
\usepackage{pdflscape} %%Adds PDF sup­port to the land­scape en­vi­ron­ment of pack­age
\usepackage{caption} %%Pro­vides many ways to cus­tomise the cap­tions in float­ing en­vi­ron­ments like fig­ure and ta­ble
\usepackage{float} %%Im­proves the in­ter­face for defin­ing float­ing ob­jects such as fig­ures and ta­bles
\usepackage[tablegrid,nochapter]{vhistory} %%Vhis­tory sim­pli­fies the cre­ation of a his­tory of ver­sions of a doc­u­ment
\usepackage[nottoc]{tocbibind} %%Au­to­mat­i­cally adds the bib­li­og­ra­phy and/or the in­dex and/or the con­tents, etc., to the Ta­ble of Con­tents list­ing
\usepackage[toc,page]{appendix} %%The ap­pendix pack­age pro­vides var­i­ous ways of for­mat­ting the ti­tles of ap­pen­dices
\usepackage{pdfpages} %%This pack­age sim­pli­fies the in­clu­sion of ex­ter­nal multi-page PDF doc­u­ments in LATEX doc­u­ments
\usepackage[rightcaption]{sidecap} %%De­fines en­vi­ron­ments called SC­fig­ure and SCtable (anal­o­gous to fig­ure and ta­ble) to type­set cap­tions side­ways
\usepackage{cite} %%The pack­age sup­ports com­pressed, sorted lists of nu­mer­i­cal ci­ta­tions, and also deals with var­i­ous punc­tu­a­tion and other is­sues of rep­re­sen­ta­tion, in­clud­ing com­pre­hen­sive man­age­ment of break points
\usepackage[]{acronym} %%This pack­age en­sures that all acronyms used in the text are spelled out in full at least once. It also pro­vides an en­vi­ron­ment to build a list of acronyms used
\usepackage[scale={.8,.8}]{geometry} %%The pack­age pro­vides an easy and flex­i­ble user in­ter­face to cus­tomize page lay­out, im­ple­ment­ing auto-cen­ter­ing and auto-bal­anc­ing mech­a­nisms so that the users have only to give the least de­scrip­tion for the page lay­out. For ex­am­ple, if you want to set each mar­gin 2cm with­out header space, what you need is just \usep­a­ck­age[mar­gin=2cm,no­head]{ge­om­e­try}.
\usepackage{layout} %%The pack­age de­fines a com­mand \lay­out, which will show a sum­mary of the lay­out of the cur­rent doc­u­ment
\usepackage{subfigure} %%Pro­vides sup­port for the ma­nip­u­la­tion and ref­er­ence of small or ‘sub’ fig­ures and ta­bles within a sin­gle fig­ure or ta­ble en­vi­ron­ment.
\usepackage[toc]{glossaries} %%The glos­saries pack­age sup­ports acronyms and mul­ti­ple glos­saries, and has pro­vi­sion for op­er­a­tion in sev­eral lan­guages (us­ing the fa­cil­i­ties of ei­ther ba­bel or poly­glos­sia).
\usepackage[left,pagewise,modulo]{lineno} %%Adds line num­bers to se­lected para­graphs with ref­er­ence pos­si­ble through the LATEX \ref and \pageref cross ref­er­ence mech­a­nism
\usepackage[colorlinks=false,hidelinks,pdfstartview=FitV]{hyperref}%%The hy­per­ref pack­age is used to han­dle cross-ref­er­enc­ing com­mands in LATEX to pro­duce hy­per­text links in the doc­u­ment. 
\usepackage{metainfo}
\usepackage[pagestyles,raggedright]{titlesec}
\usepackage{etoolbox}
\usepackage{%
	array, %%An ex­tended im­ple­men­ta­tion of the ar­ray and tab­u­lar en­vi­ron­ments which ex­tends the op­tions for col­umn for­mats, and pro­vides "pro­grammable" for­mat spec­i­fi­ca­tions
	booktabs, %%The pack­age en­hances the qual­ity of ta­bles in LATEX, pro­vid­ing ex­tra com­mands as well as be­hind-the-scenes op­ti­mi­sa­tion
	dcolumn, %%
	rotating,
	shortvrb,
	units,
	url,
	lastpage,
	longtable,
	lscape,
	qtree,
	skmath,
	tabu
}
%% Java --> latex 
\usepackage{listings}
\usepackage{color}
\definecolor{pblue}{rgb}{0.13,0.13,1}
\definecolor{pgreen}{rgb}{0,0.5,0}
\definecolor{pred}{rgb}{0.9,0,0}
\definecolor{pgrey}{rgb}{0.46,0.45,0.48}
\newcommand\tab[1][1cm]{\hspace*{#1}}
\usepackage{inconsolata}
%%Listing style for java.
\definecolor{dkgreen}{rgb}{0,0.6,0}
\definecolor{gray}{rgb}{0.5,0.5,0.5}
\definecolor{mauve}{rgb}{0.58,0,0.82}
\lstset{frame=tb,
	language=Java,
	aboveskip=3mm,
	belowskip=3mm,
	showstringspaces=false,
	columns=flexible,
	basicstyle={\small\ttfamily},
	numbers=left,
	numberstyle=\tiny\color{gray},
	keywordstyle=\color{blue},
	commentstyle=\color{dkgreen},
	stringstyle=\color{mauve},
	breaklines=true,
	breakatwhitespace=true,
	tabsize=3
}


\setlength{\parindent}{0pt}
\setlength{\parskip}{.5\baselineskip}

%% Inserting the metadata

% % Metadata for Document
\def\Company{FPS Gaming}
\def\Institute{\textit{Maulana Abul Kalam Azad University of Technology, West Bengal}}
\def\Course{\textit{Computer Science and Engineering}}
\def\Module{\textit{Software Engineering}}
\def\Docent{\textit{Dr. Suparna Biswas (Saha)}}
\def\Assistant{\textit{}}

\def\BoldTitle{Software Requirements Specification}

\def\Subtitle{for \\ Shadows (FPS game) \\}
\def\Authors{Prepared by \\\\ Debayan De(30000114010) \\\\ Rohit Das(30000114022) \\\\ Rudra Nil Basu(30000114023) \\\\ Sumitro Chowdhury(30000114027) } 
\def\Shortname{}


\title{\textbf{\BoldTitle}\\\Subtitle}
\author{\Authors \\ \\ \\ \Institute\\ \Course\\ \Module\\ \Docent\\}
\date{Kolkata, 8. November, 2017}

%%%%%%%%%%%%%%%%%%%%%%%%%%%%%%%%%%%%%%%%%%%%%%%%%%%%%%%%%%%%%%%%%%%%%%%%%%%%%%%%%
%% Creation of pdf information
%%%%%%%%%%%%%%%%%%%%%%%%%%%%%%%%%%%%%%%%%%%%%%%%%%%%%%%%%%%%%%%%%%%%%%%%%%%%%%%%%
\hypersetup{pdfinfo={
		Title={Title},
		Author={TR},
		Subject={Report}
	}}

%% Creating the frontpage

\AtBeginDocument{
	\maketitle
	\thispagestyle{empty}
}

%% Creation of the header

\patchcmd{\chapter}{plain}{short}{}{} %$ <-- the header on chapter 1

%% Creation of page-styles

\newpagestyle{long}{%
	\sethead[\thepage][][\chaptername\ \thechapter:\ \chaptertitle]{\chaptername\ \thechapter:\ \chaptertitle}{}{\thepage}
	\headrule
}

\newpagestyle{short}{%
	\sethead[\thepage][][]{}{}{\thepage}
	\headrule
}

%% DOCUMENT
\begin{document}

\pagenumbering{roman}
\DeclareGraphicsExtensions{.pdf,.jpg,.png}
\pagestyle{short}

\newpage
%% Table of contents
\Large \tableofcontents % Inhaltsverzeichnis

\pagestyle{long}

%% Version table insertion
%% Versionstabelle.

\chapter*{Revision History}
\addcontentsline{toc}{chapter}{Revision History}
\begin{versionhistory}
	\vhEntry{1.0}{25.09.2016}{A.Sandu}{Chapter 1 - Introduction}
    \vhEntry{2.0}{10.12.2015}{}{Explaining different sortings}
    \vhEntry{3.0}{05.01.2016}{}{Kleine Änderungen}
    \vhEntry{4.0}{10.01.2016}{}{Finale Version}


\end{versionhistory}
\pagenumbering{arabic}
%% Inserting all the content
\Large \chapter{Introduction}
\label{ch:intro}
\tab{\Large This section gives a scope description and overview of everything included in this SRS document. Also the purpose of this document is described and a list of abbreviations and definitions is provided.}
\section{Purpose}
\tab{The purpose of this document is to give a detailed description of the requirements of the "“SHADOWS“" software (game).It will illustrate the purpose and complete declaration for the development of system. It will also explain constraints, interface and interactions with users and external applications. This document is primary intended to be proposed to a customer for its approval and a reference for describing the first version of the system for the development team.}
\section{Document Conventions}
%\begin{table}
	%\resziebox{\textwidth}{!}{%
	\begin{tabu} to 1.0\textwidth{|X[l]|X[r]|}
		\hline
		Term & Definitions\\
		\hline\hline
		Users/players & Someone who interacts with the system\\
		\hline
		Fps & First Person Shooter\\
		\hline
	\end{tabu}
%\end{table}
%\section{Intended Audience and Reading Suggestions}
\section{Product Scope}
\tab{"SHADOWS"” is a multiplayer, fps, horror game mainly designed for people with greater than 16 years .It is for entertainment purpose only. Players can join by lobby and play game.}
\section{References}
{\begin{enumerate}
	\item https://unity3d.com/learn/tutorials
	\item https://www.assetstore.unity3d.com/en/\#
\end{enumerate}}

\chapter{Overall Description}
\label{Overall Description}
\tab{\Large This section will give an overview of the whole system. The system will be explained in its context to show how the system interacts with another system and introduce the basic functionality of it. It will also describe what type of stakeholders that will use the system and what functionality is available for each type. At last, the constraints and assumptions for the system will be presented. }

\section{Product Perspective}

\section{Product Functions}
\tab{With this application from the menu driven GUI interface can play game or quit game. He can create a lobby with other players and play game using keyboard and mouse. KEY FEATURES OF THE GAME PLZ MENTION.}

\section{User Classes and Characteristics}
\tab{There are only one type of users in this system, the players. Players interact with the system with the help of GUI interface.}

\section{Operating Environment}

\section{Design and Implementation Constraints}
\tab{The network connection is a constraint for the application. Since the application fetches data from the database over the internet. It is crucial that there is an Internet connection for the application to function.}\\
\tab {Power failure is also a constraint, as there is a multiplayer game, so there is no saving option.}

\section{User Documentation}
\tab{This product is under development state and requires a complete implemented prototype to explain the user documentation.}

\section{Assumptions and Dependencies}
\tab{Our assumption about the product is that it will always be used on Windows OS and pc having enough capability to run this application smoothly.}\\
\tab{Another assumption is that there is always a good Internet connection.}

\newpage




\chapter{Specific Requirements}
\label{Specific Requirements}
\tab{\Large This section contains all of the functional and quality requirements of the system. It gives a detailed description of the system and all its features.}
\section{External Interface Requirements}
\tab{This section provides a detailed description of all inputs into and outputs from the system. It also gives a description of the hardware, software and communication interfaces and provides basic prototypes of the user interface.}
\subsection{User Interfaces}
\tab{RNB DO IT PHOTO LAGBE AR DESCRIPTION}
\subsection{Hardware Interfaces}
\tab{Since the application does not have any designated hardware. It does not have any direct hardware interface.}
\subsection{Software Interfaces}
\tab{It uses mostly C\# language for coding, Unity game engine for developing the game and Blender for graphical models.	}
%\section{Communications Interfaces}


%\chapter{System Features}
%\label{System Features}

%\section{System Feature 1}
\section{Functional Requirements}

\section{Other Nonfunctional Requirements}
%\label{Other Nonfunctional Requirements}

\subsection{Performance Requirements}
\subsection{Safety Requirements}
\subsection{Security Requirements}
%\section{Software Quality Attributes}
%\section{Business Rules}

\section{Other Requirements}
%\label{Other Requirements}
\subsection{Legal Requirements}

\begin{appendices}
\chapter{Glossary}
\chapter{Analysis Models}
\chapter{To Be Determined List}


\end{appendices}




%% Source defintions
% When no use outcomment
%\bibliographystyle{alpha}

\renewcommand\bibname{References}
\bibliography{base/sources}


%% Inserting the appendix
% When no use outcomment
%\newpage
\appendix 
% Adds appendix as chapter to toc
\addcontentsline{toc}{chapter}{Appendix}


\chapter{First}


\chapter{Second}


\end{document}*/***********************************************************************8	
